\documentclass[11pt]{exam}
\newcommand{\myname}{Ling Song, Hosein Hadipour} %Write your name in here
\newcommand{\myhwtype}{Homework}
\newcommand{\myhwnum}{1} %Homework set number
\newcommand{\myclass}{Cryptanalysis 2021}
\newcommand{\mylecture}{}
\newcommand{\mysection}{}

% Prefix for numedquestion's
\newcommand{\questiontype}{Question}

% Use this if your "written" questions are all under one section
% For example, if the homework handout has Section 5: Written Questions
% and all questions are 5.1, 5.2, 5.3, etc. set this to 5
% Use for 0 no prefix. Redefine as needed per-question.
\newcommand{\writtensection}{0}

\usepackage{amsmath, amsfonts, amsthm, amssymb}  % Some math symbols
\usepackage{enumerate}
\usepackage{enumitem}
\usepackage{graphicx}
\usepackage{hyperref}
\usepackage[all]{xy}
\usepackage{wrapfig}
\usepackage{fancyvrb}
\usepackage[T1]{fontenc}
\usepackage{listings}

\usepackage{centernot}
\usepackage{mathtools}
\DeclarePairedDelimiter{\ceil}{\lceil}{\rceil}
\DeclarePairedDelimiter{\floor}{\lfloor}{\rfloor}
\DeclarePairedDelimiter{\card}{\vert}{\vert}


\setlength{\parindent}{0pt}
\setlength{\parskip}{5pt plus 1pt}
\pagestyle{empty}

\def\indented#1{\list{}{}\item[]}
\let\indented=\endlist

\newcounter{questionCounter}
\newcounter{partCounter}[questionCounter]

\newenvironment{namedquestion}[1][\arabic{questionCounter}]{%
    \addtocounter{questionCounter}{1}%
    \setcounter{partCounter}{0}%
    \vspace{.2in}%
        \noindent{\bf #1}%
    \vspace{0.3em} \hrule \vspace{.1in}%
}{}

\newenvironment{numedquestion}[0]{%
	\stepcounter{questionCounter}%
    \vspace{.2in}%
        \ifx\writtensection\undefined
        \noindent{\bf \questiontype \; \arabic{questionCounter}. }%
        \else
          \if\writtensection0
          \noindent{\bf \questiontype \; \arabic{questionCounter}. }%
          \else
          \noindent{\bf \questiontype \; \writtensection.\arabic{questionCounter} }%
        \fi
    \vspace{0.3em} \hrule \vspace{.1in}%
}{}

\newenvironment{alphaparts}[0]{%
  \begin{enumerate}[label=\textbf{(\alph*)}]
}{\end{enumerate}}

\newenvironment{arabicparts}[0]{%
  \begin{enumerate}[label=\textbf{\arabic{questionCounter}.\arabic*})]
}{\end{enumerate}}

\newenvironment{questionpart}[0]{%
  \item
}{}

\newcommand{\answerbox}[1]{
\begin{framed}
\vspace{#1}
\end{framed}}

\pagestyle{head}

\headrule
\header{\textbf{\myclass\ \mylecture\mysection}}%
{\textbf{\myname}}%
{\textbf{\myhwtype\ \myhwnum}}

\begin{document}
\thispagestyle{plain}
\begin{center}
  {\Large \myclass{} \myhwtype{} \myhwnum} \\
  \myname{}\\
  \today
\end{center}


%Here you can enter answers to homework questions

\begin{numedquestion}
Consider the following structure where ENC and DEC represent the encryption and decryption via the \texttt{DES} algorithm with 56-bit key respectively. Note that, the first and last encryption blocks use the same 56-bit key $k_{1}$, whereas the middle one utilizes $k_{2}$ which is not necessarily the same as $k_{1}$. Does it provide a $112$ bits security level? If not so, provide a cryptographic attack with time complexity of strictly less than $2^{112}$ \texttt{DES} encryptions. Please explain what model is your attack classified in (known-plaintext, chosen-plaintext, $\ldots$). Besides, the amount of \textbf{time} and \textbf{memory} in your attack should be specified in detail.

\textit{A known plaintext attack has more points in comparison to a chosen plaintext attack.}
\begin{figure}[ht!]
	\begin{center}
		\includegraphics[scale=0.5]{./Figures/two_key_triple_encryption.PNG}
	\end{center}\caption{Two-Key Triple Encryption} \label{fig:two_key_three_encryptions}
\end{figure}
\end{numedquestion}

\begin{numedquestion}
This concerns using memory in favor of speed in implementing the encryption algorithms. Let $S$ be the AES S-box. Let $MC$ be the mix-column, i.e., it takes as input a column consisting of 4 bytes and outputs such a column. Note that $MC$ is linear, i.e., we have for any two columns $C_{1}, C_{2}$ that:
\begin{equation*}
	MC(C_{1} \oplus C_{2}) = MC(C_{1})\oplus MC(C_{2}). 
\end{equation*}
 
 Define a function $T_{0}$ as follows: it takes as input a byte $b$, and the output $T_{0}(b)$ is a 4-byte column computed as follows: you first form a column $C(b)$ by placing $S(b)$ in the top byte and all-0 bytes in the lower three positions. Then set $T_{0}(b) = MC(C(b))$. We also define functions $T_{1}, T_{2}, T_{3}$. They are similar to $T_{0}$, except that when we form $C(b)$, we place $S(b)$ in the second, third and fourth entry from the top respectively, and put in 0's elsewhere. 
 
 Now consider the state of AES encryption algorithm at the start of some round. Name the bytes in this state $a_{ij}$ as in ? and let $R$ be the state after we have done \texttt{SubBytes}, \texttt{ShiftRow} and \texttt{MixColumn}. So $R$ is a 4 by 4 matrix of bytes. 
 
 Show that the first column of $R$ is
 \begin{equation*}
T_{0}(a_{00})\oplus T_{1}(a_{11}) \oplus T_{2}(a_{22}) \oplus T_{3}(a_{33}).
 \end{equation*}
Give similar expressions for the other 3 columns of $R$. 

Sketch how this result can be used to implement AES based only on table look-up and XOR, instead of explicitly computing the operations. How much memory would you need for this?  

\begin{figure}[ht!]
	\begin{center}
		\includegraphics[scale=1.5]{./Figures/aes_round_function.pdf}
	\end{center}\caption{A Round of AES} \label{fig:aes_round_function}
\end{figure}

\end{numedquestion}


% if you do not solve some of the questions use this command to increment counter
% \setcounter{questionCounter}{4}
\begin{numedquestion}
Let $S = \{1, 2, \ldots, d\}$, where $d\in \mathbb{N}$. Besides, let $A = \{a_{1}, \ldots, a_{m}\}$ and $B = \{b_{1}, \ldots, b_{n}\}$ be two subsets of $S$. 
  \begin{alphaparts}
	\item Prove that if $m\times n \geq d$, then $\Pr\{A \cap B \neq \emptyset\} \geq 0.5.$
	\item Assume that $f$ is a function from $S = \{1, \ldots, N\}$ to itself where $N\in \mathbb{N}$. Besides, let $x_{1}^{0}, \ldots, x_{m}^{0}$ be some elements of $S$. Build the following array of chains: 
	\begin{align*}
		&x_{1}^{0} \longrightarrow x_{1}^{1} = f(x_{1}^{0}) \longrightarrow & \cdots & \longrightarrow x_{1}^{t - 1} = f(x_{1}^{t - 2})\\
		&x_{2}^{0} \longrightarrow x_{2}^{1} = f(x_{1}^{0}) \longrightarrow & \cdots & \longrightarrow x_{1}^{t - 1} = f(x_{2}^{t - 2})\\
		&\cdots\\
		&x_{m}^{0} \longrightarrow x_{m}^{1} = f(x_{1}^{0}) \longrightarrow & \cdots & \longrightarrow x_{m}^{t - 1} = f(x_{m}^{t - 2}),\\
	\end{align*}
where $t\in \mathbb{N}$. Prove that if $mt^{2} \leq N$, then all entries in the above array will be different with high probability ($\Pr \geq 0.5$).
    \item  To do
\end{alphaparts}



\end{numedquestion}

\begin{numedquestion}
This concerns a trick that is very useful to find a cycle in a sequence of iterated function values. Let $S$ be any finite set, $f$ be any function from $S$ to itself, and $x_{0}$ be any element of $S$. For any $i > 0$, let $x_{i} = f(x_{i - 1})$. Let $\mu$ be the smallest index such that the value $x_{\mu}$ reappears infinitely often within the sequence of values $x_{i}$, and let $\lambda$ be the smallest positive integer such that $x_{\mu} = x_{\lambda + \mu}$. 
\begin{alphaparts}
\item 
Prove that $i = k\lambda \geq \mu$ for some $k$ if and only if $x_{i} = x_{2i}$. 
\item Based on the above fact, propose an algorithm to find $\mu$ and $\lambda$, given $f$ and $x_{0}$. 
\item Using the proposed algorithm in the previous part, find a 64-bit collision for $\texttt{SHA3-512}$. 
\end{alphaparts}
Hint: Study about the cycle detection algorithms in \url{https://en.wikipedia.org/wiki/Cycle_detection}.

To compute the $\texttt{SHA3-512}$ using the \texttt{Python} language you can use the following commands: 
\begin{lstlisting}[language=Python]
In [1]: import hashlib
In [2]: st = "Hello World!"
In [3]: digest = hashlib.sha3_512(st.encode())
In [4]: digest.hexdigest()
Out[4]: '32400b5e89822de254e8d5d94252c52bdcb27a3562ca593e980364d9848b8041
b98eabe16c1a6797484941d2376864a1b0e248b0f7af8b1555a778c336a5bf48'
\end{lstlisting}
\end{numedquestion}
\end{document}